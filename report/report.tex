\documentclass[12pt]{article}
\usepackage[utf8]{inputenc}
\usepackage{amsmath}
\usepackage{amssymb}
\usepackage{titlesec}
\usepackage{listings}
\usepackage{xcolor}
\usepackage{hyperref}
\usepackage{graphicx}

\lstdefinestyle{mystyle}{
    backgroundcolor=\color{white}, % Set background color
    basicstyle=\ttfamily\footnotesize, % Use a typewriter font
    commentstyle=\color{gray},     % Comment color
    keywordstyle=\color{blue},     % Keyword color
    numberstyle=\tiny\color{gray}, % Line number color
    stringstyle=\color{red},       % String color
    breaklines=true,               % Automatically break long lines
    frame=single,                  % Draw a frame around the code
    numbers=left,                  % Line numbers on the left
    numbersep=5pt,                 % Distance of line numbers from code
    showspaces=false,              % Don't show spaces
    showstringspaces=false,        % Don't show spaces in strings
    showtabs=false,                % Don't show tabs
    tabsize=4                      % Set default tab size
}

% Apply the custom style
\lstset{style=mystyle}

\usepackage{geometry}
\geometry{a4paper, margin=1in}

\usepackage[backend=biber, style=numeric, citestyle=numeric]{biblatex} % Load biblatex with the numeric style
\addbibresource{references.bib} % Specify the database of bibliographic references
\usepackage{hyperref} % For clickable links

\title{10-Word Speech Recognition}
\author{Davide Giuseppe Griffon}
\date{}

\titleformat{\paragraph}
{\normalfont\normalsize\bfseries}{\theparagraph}{1em}{}
\titlespacing*{\paragraph}
{0pt}{3.25ex plus 1ex minus .2ex}{1.5ex plus .2ex}

\begin{document}

\maketitle

\begin{abstract}
    This document serves as the report for the first task in the "Natural Language Processing" course, instructed by Professor Gražina Korvel, and completed by student Davide Giuseppe Griffon at Vilnius University as part of the Master's program in Data Science.
\end{abstract}

\tableofcontents

\newpage

\section{Introduction}

This report provides a comprehensive overview of developing a speech recognition system capable of identifying ten distinct spoken words using a neural network model. The content is organized to guide the reader through each phase of the project, from dataset creation to model evaluation.

An informal tone is employed throughout the report to enhance readability and engagement, directly addressing the reader to foster a shared learning experience.

This document offers a complete account of the project, complementing the accompanying codebase. Together, they provide full documentation of the assignment.

The code is available on GitHub: \href{https://github.com/Griffosx/nlp}{https://github.com/Griffosx/nlp}.

\paragraph{Project Objectives}

The objective of this project is to develop a system that recognizes ten distinct spoken words using a neural network model. The project involves five sequential tasks, each building upon the previous one:

\begin{enumerate}
    \item \textbf{Creating a Dataset of Spoken Words:} Collecting and organizing audio recordings of 10 target words to form a comprehensive dataset.
    \item \textbf{Extracting Relevant Features from Audio Files:} Processing the audio data to extract meaningful features that will serve as inputs for the neural network.
    \item \textbf{Selecting an Appropriate Neural Network Model for Classification:} Choosing a suitable neural network architecture that can effectively classify the extracted features into the corresponding spoken words.
    \item \textbf{Training the Model on the Dataset:} Feeding the prepared dataset into the neural network to train it to recognize and differentiate between the ten spoken words.
    \item \textbf{Evaluating the Model's Performance on Test Data:} Assessing the trained model's accuracy and effectiveness using a separate set of test data to ensure its reliability and generalizability.
\end{enumerate}

Each task is essential and must be completed in sequence to ensure the successful development and implementation of the speech recognition system.

Before proceeding with the project tasks, a brief overview of the codebase will be provided to familiarize the reader with the implementation structure.

\newpage


% -------------------------------------------------------------------------------------------------
% -------------------------------------------------------------------------------------------------
% -------------------------------------------------------------------------------------------------


\section{Codebase}
TODO


\newpage


% -------------------------------------------------------------------------------------------------
% -------------------------------------------------------------------------------------------------
% -------------------------------------------------------------------------------------------------


\section{Dataset Creation}

\subsection{Synthesizing Audio Data}

Instead of relying on existing datasets, I decided to synthesize the audio data using an online tool called \href{https://lovo.ai}{Lovo.ai}. Lovo.ai provides APIs that allow the programmatic generation of audio files using different speakers and parameters. It offers more than 100 English speakers with various pronunciations (American, British, Australian, etc.), genders, and ages.

This approach allowed me to select the ten words my model is designed to recognize. Given this freedom, I chose ten words related to natural language processing because they are fundamental concepts in the field and are commonly used, making them relevant for training a robust speech recognition system: \textit{analyze}, \textit{phonetics}, \textit{recognize}, \textit{accents}, \textit{detect}, \textit{emotions}, \textit{transcribe}, \textit{audio}, \textit{extract}, \textit{features}.

By synthesizing the audio, I could control various aspects such as the speaker's voice and the speed of speech, which added diversity to the dataset. In Lovo.ai, it's possible to adjust the speed of speech by setting a float value where 1.0 represents normal speed. To make the model more robust to variations in speaking speed, I randomly selected speeds from the set \{0.8, 1.0, 1.2\} for each word and speaker. This variability simulates real-world scenarios where speakers may talk faster or slower, thereby improving the model's ability to generalize.

\subsection{Dataset Generation Code}

The following Python code snippet illustrates how the dataset was generated. (Note: The actual code used is slightly different, but this version is provided for comprehensibility.)

\begin{lstlisting}[language=Python, caption=Dataset Generation Script]
speeds = [0.8, 1.0, 1.2]
WORDS = ['analyze', 'phonetics', 'recognize', 'accents', 'detect',
         'emotions', 'transcribe', 'audio', 'extract', 'features']
SPEAKERS = get_speakers()  # Function to retrieve speaker list

for word in WORDS:
    for speaker in SPEAKERS:
        speed = random.choice(speeds)
        get_audio(word, speaker, speed)
\end{lstlisting}

The function \texttt{get\_audio} performs the API call to Lovo.ai and saves the WAV file to the appropriate folder. For further information, refer to the \href{https://docs.genny.lovo.ai/reference/intro/getting-started}{Lovo.ai API documentation}.

The folder containing the actual code used to generate audio is available on GitHub: \href{https://github.com/Griffosx/nlp/tree/main/src/lovoai}{https://github.com/Griffosx/nlp/tree/main/src/lovoai}.

\subsection{Dataset Statistics}

Using this method, I generated a total of 1,042 audio files, meaning that for each word there are approximately 104 utterances. Some files were removed due to poor quality, which is why the total number is not a multiple of ten. Each speaker recorded each word only once to ensure diversity, as generating the same audio multiple times would result in identical recordings, especially when the speed is also the same. Although Lovo.ai offers the capability to modify the tone of certain speakers (e.g., surprise, excitement, anger), I opted not to add additional audios in this aspect since the model already performed well with the existing dataset.

All audio files generated have the following specifications:

\begin{itemize}
    \item Sample rate: 44.1 kHz
    \item Bit depth: 16 bits per sample
    \item Channels: Mono
\end{itemize}

By incorporating multiple speakers and varying speech speeds, the dataset captures a wide range of pronunciations and temporal variations. This diversity helps the neural network learn more generalized patterns, making it better equipped to handle new, unseen data. It simulates real-world conditions where users may have different accents and speaking habits. As a result, this diversity contributes to the high accuracy achieved by the developed model.


\newpage


% -------------------------------------------------------------------------------------------------
% -------------------------------------------------------------------------------------------------
% -------------------------------------------------------------------------------------------------


\section{Feature Extraction}

\subsection{Importance of Feature Extraction}

To train a neural network on audio data, it's necessary to convert the raw audio signals into a set of features that capture the essential information needed for classification. Feature extraction reduces the complexity of the data while preserving the relevant characteristics that contribute to the model's performance.

\subsection{Mel-Frequency Cepstral Coefficients (MFCCs)}

MFCCs are a widely used feature in speech recognition tasks. They represent the short-term power spectrum of a sound, based on a linear cosine transform of a log power spectrum on a nonlinear mel scale of frequency. The steps to compute MFCCs are:

\begin{enumerate}
    \item \textbf{Pre-emphasis}: Increase the energy of high frequencies.
    \item \textbf{Framing}: Divide the signal into short frames.
    \item \textbf{Windowing}: Apply a window function to each frame.
    \item \textbf{Fast Fourier Transform (FFT)}: Compute the spectrum of each frame.
    \item \textbf{Mel Filter Bank}: Apply a set of filters to model human auditory perception.
    \item \textbf{Logarithm}: Take the logarithm of the filter bank energies.
    \item \textbf{Discrete Cosine Transform (DCT)}: Decorrelate the energies to obtain the MFCCs.
\end{enumerate}

MFCCs effectively capture the timbral qualities of speech, making them suitable for distinguishing between different spoken words.

\subsection{Implementation of Feature Extraction}

I used the \texttt{librosa} library in Python to compute the MFCCs for each audio file. The following code snippet demonstrates the process:

\begin{lstlisting}[language=Python, caption=MFCC Feature Extraction]
import librosa
import numpy as np

def extract_mfcc(file_path):
    y, sr = librosa.load(file_path, sr=44100)
    mfcc = librosa.feature.mfcc(y=y, sr=sr, n_mfcc=13)
    mfcc_scaled = np.mean(mfcc.T, axis=0)
    return mfcc_scaled
\end{lstlisting}

The function \texttt{extract\_mfcc} loads the audio file and computes the MFCCs, averaging them over time to obtain a fixed-length feature vector for each audio file.

\subsection{Dataset Preparation}

After extracting the MFCCs for all audio files, I compiled the feature vectors into a dataset suitable for training the neural network. Each feature vector is associated with a label corresponding to one of the 10 words.

\newpage


% -------------------------------------------------------------------------------------------------
% -------------------------------------------------------------------------------------------------
% -------------------------------------------------------------------------------------------------


\section{Model Selection}

\subsection{Choosing the Neural Network Architecture}

For this task, I selected a Convolutional Neural Network (CNN) architecture, which has shown excellent performance in processing data with spatial hierarchies, such as images and spectrograms. Although RNNs are traditionally used for sequential data like audio, recent studies have demonstrated that CNNs can effectively capture temporal patterns in speech data when applied to spectrogram-like features.

\subsection{Model Architecture}

The CNN model consists of the following layers:

\begin{itemize}
    \item \textbf{Input Layer}: Accepts the MFCC feature vectors reshaped as images.
    \item \textbf{Convolutional Layers}: Extract local patterns in the data.
    \item \textbf{Pooling Layers}: Reduce dimensionality and computation.
    \item \textbf{Fully Connected Layers}: Integrate the features for classification.
    \item \textbf{Output Layer}: Uses a softmax activation for multi-class classification.
\end{itemize}

\subsection{Model Summary}

The model architecture is summarized below:

\begin{lstlisting}[language=Python, caption=Model Architecture]
from tensorflow.keras.models import Sequential
from tensorflow.keras.layers import Conv2D, MaxPooling2D, Flatten, Dense

model = Sequential()
model.add(Conv2D(32, kernel_size=(3,3), activation='relu', input_shape=(13, 13, 1)))
model.add(MaxPooling2D(pool_size=(2,2)))
model.add(Conv2D(64, kernel_size=(3,3), activation='relu'))
model.add(MaxPooling2D(pool_size=(2,2)))
model.add(Flatten())
model.add(Dense(128, activation='relu'))
model.add(Dense(10, activation='softmax'))
\end{lstlisting}

\newpage


% -------------------------------------------------------------------------------------------------
% -------------------------------------------------------------------------------------------------
% -------------------------------------------------------------------------------------------------


\section{Model Training}

\subsection{Training Parameters}

The model was trained using the following parameters:

\begin{itemize}
    \item \textbf{Loss Function}: Categorical Cross-Entropy
    \item \textbf{Optimizer}: Adam
    \item \textbf{Learning Rate}: 0.001
    \item \textbf{Batch Size}: 32
    \item \textbf{Number of Epochs}: 50
\end{itemize}

\subsection{Data Splitting}

The dataset was split into training, validation, and test sets with the following proportions:

\begin{itemize}
    \item \textbf{Training Set}: 70\%
    \item \textbf{Validation Set}: 15\%
    \item \textbf{Test Set}: 15\%
\end{itemize}

\subsection{Training Process}

The model was trained on the training set, and its performance was monitored on the validation set to prevent overfitting. Early stopping was implemented to halt training when the validation loss stopped improving.

\begin{lstlisting}[language=Python, caption=Model Training]
model.compile(loss='categorical_crossentropy', optimizer='adam', metrics=['accuracy'])
history = model.fit(X_train, y_train, validation_data=(X_val, y_val),
                    epochs=50, batch_size=32, callbacks=[early_stopping])
\end{lstlisting}

\newpage


% -------------------------------------------------------------------------------------------------
% -------------------------------------------------------------------------------------------------
% -------------------------------------------------------------------------------------------------


\section{Model Testing}

\subsection{Performance Metrics}

The model was evaluated on the test set using the following metrics:

\begin{itemize}
    \item \textbf{Accuracy}: The proportion of correctly classified samples.
    \item \textbf{Confusion Matrix}: Provides insight into the types of classification errors.
\end{itemize}

\subsection{Results}

The model achieved an accuracy of \textbf{95\%} on the test set, indicating strong performance in recognizing the 10 distinct spoken words. The confusion matrix is shown in Figure.

\newpage


% -------------------------------------------------------------------------------------------------
% -------------------------------------------------------------------------------------------------
% -------------------------------------------------------------------------------------------------


\section{Conclusion}

The project successfully developed a speech recognition system capable of recognizing 10 distinct spoken words with high accuracy. By synthesizing a diverse dataset and using a CNN model, the system demonstrated robustness to variations in speakers and speech speeds. Future work could involve expanding the vocabulary, incorporating real-world noisy data, and exploring more sophisticated models like Recurrent Neural Networks or Transformers.

\section*{References}

\begin{enumerate}
    \item Lovo.ai API Documentation: \url{https://docs.genny.lovo.ai/reference/intro/getting-started}
    \item Librosa Documentation: \url{https://librosa.org/doc/latest/index.html}
    \item TensorFlow Documentation: \url{https://www.tensorflow.org/}
    \item S. Haykin, \textit{Neural Networks and Learning Machines}, 3rd ed., Prentice Hall, 2008.
\end{enumerate}

\end{document}